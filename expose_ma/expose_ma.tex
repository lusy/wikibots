% ---------------------------------------------------
% ----- Main document of the template
% ----- for Bachelor-, Master thesis and class papers
% ---------------------------------------------------
%  Created by Claudia Müller-Birn on 2012-08-17. (last update 2015-10-20)
%  Freie Universität Berlin, Institute of Computer Science, Human Centered Computing (HCC).
%
\documentclass[pdftex,a4paper,11pt]{scrartcl}
%
%---------------------------------------------------
%----- Packages
%---------------------------------------------------
%
\usepackage[T1]{fontenc}
\usepackage[utf8]{inputenc}
%\usepackage[ngerman]{babel}
\usepackage[english]{babel}
\usepackage{ae}
\usepackage{bibgerm}

\usepackage{fancyref}
\usepackage{fancyhdr} % Define simple headings
\usepackage{xcolor}
\usepackage{url}
\usepackage{verbatim}
%
\usepackage[pdftex]{graphicx}
\usepackage{hyperref} % turn all your internal references into hyperlinks
%\usepackage[pdfstartview=FitH,pdftitle={<<Titel der Arbeit>>}, pdfauthor={<<Autor>>}, pdfkeywords={<<Schlüsselwörter>>}, pdfsubject={<<Titel der Arbeit>>}, colorlinks=true, linkcolor=black, citecolor=black, urlcolor=black, hypertexnames=false, bookmarksnumbered=true, bookmarksopen=true, pdfborder = {0 0 0}]{hyperref}
%
% a new command is defined that allows to include an empty page when needed
\newcommand{\blankpage}{
\newpage
\thispagestyle{empty}
\mbox{}
\newpage
}
%
%---------------------------------------------------
%----- PDF and document setup
%---------------------------------------------------
%
\hypersetup{
	pdftitle={<My title>},  % please, add the title of your thesis
    pdfauthor={<Author>},   % please, add your name
    pdfsubject={<<Bachelor/Master thesis>, Institute of Computer Science, Freie Universität Berlin>}, % please, select the type of this document
    pdfstartview={FitH},    % fits the width of the page to the window
    pdfnewwindow=true, 		% links in new window
    colorlinks=false,  		% false: boxed links; true: colored links
    linkcolor=red,          % color of internal links
    citecolor=green,        % color of links to bibliography
    filecolor=magenta,      % color of file links
    urlcolor=cyan           % color of external links
}
%
%---------------------------------------------------
%----- Settings for word separation
%---------------------------------------------------
% Help for separation (from package babel, section 22)):
% In german package the following hints are additionally available:
% "- = an explicit hyphen sign, allowing hyphenation in the rest of the word
% "| = disable ligature at this position. (e.g., Schaf"|fell)
% "~ = for a compound word mark without a breakpoint (e.g., bergauf und "~ab)
% "= = for a compound word mark with a breakpoint, allowing hyphenation in the composing words
% "" = like "-, but producing no hyphen sign (e.g., und/""oder)
%
% Describe separation hints here:
\hyphenation{
% Pro-to-koll-in-stan-zen
% Ma-na-ge-ment  Netz-werk-ele-men-ten
% Netz-werk Netz-werk-re-ser-vie-rung
% Netz-werk-adap-ter Fein-ju-stier-ung
% Da-ten-strom-spe-zi-fi-ka-tion Pa-ket-rumpf
% Kon-troll-in-stanz
}

%---------------------------------------------------
%----- Settings for title page
%---------------------------------------------------

\begin{titlepage}

\title{\includegraphics[width=0.6\textwidth]{pics/FU_logo.pdf}\\
{\small <Bachelor-/Masterarbeit> am Institut für Informatik der Freien Universität Berlin}\\
{\small Human-Centered Computing (HCC)}\\
[6ex]
{\LARGE<Titel der Arbeit>}\\
{\normalsize-- Exposé --}}

\author{
{\emph{\normalsize<Ihr Vor- und Nachname>}}\\
{\normalsize Matrikelnummer: <IhreMatrikelnummer>}\\
{\normalsize <ihreemail@adresse.de>}\\\\
{\normalsize Betreuerin: Prof. Dr. C. Müller-Birn}
}

\date{\normalsize Berlin, <Datum>}

\end{titlepage}

%%%%%%%%%%%%%%%%%%%%%%%%%%%%%%%%%%%%%%%%%%%%%%%%%%%%%%
% The content part of this document starts here! %%
%%%%%%%%%%%%%%%%%%%%%%%%%%%%%%%%%%%%%%%%%%%%%%%%%%%%%%

\begin{document}

\maketitle

\thispagestyle{empty}  % remove page number on the title page

\blankpage

%---------------------------------------------------
%----- Content part
%---------------------------------------------------
\setcounter{page}{1} % page number is set to "1" otherwise it would be "3"

\section{Struktur des Exposé}
\begin{comment}
Im Folgenden habe ich Ihnen eine generelle Struktur für ein Exposé vorgegeben. Jeder Abschnitt wird mit Fragen eingeleitet, welchen den Inhalt des Abschnitts abdecken. Ich gebe Ihnen, wenn erforderlich, noch einige Erläuterungen. Bitte beachten Sie, dass das Layout dieser Vorlage doppelseitig angelegt ist und daher auch der Ausdruck doppelseitig erfolgen sollte. Zur Erinnerung: ein Exposé umfasst ungefähr sechs bis zehn Seiten abhängig vom Thema. Der Inhalt des Exposé bildet dann das Grundgerüst für die schriftliche Ausarbeitung Ihrer Masterarbeit. Erst nach Abnahme des Exposé sollten Sie Ihre Masterarbeit anmelden.
\end{comment}

\subsection{Motivation}
\noindent \emph{In welchem Bereich/Themenfeld bewegt sich Ihre geplante Arbeit? }

This thesis is ubiquited in the field of human-computer interaction and more
specifically, algorithmic governance and socio-technical systems.
Algorithmic governance refers to the notion of algorithms used in governance systems, (partially) taking decisions concerning humans.
As Danaher et al. put it ``algorithms are increasingly being used to nudge, bias, guide, provoke, control, manipulate and constrain human behaviour''~\cite{DanaherEtAl2017}.
%TODO: def!

Wikipedia is a complex socio-technical system where mechanisms of algorithmic governance are increasingly used.
Since its birth in 2001, the free online encyclopedia has come to comprise nearly 300 different language versions and 48,364,313 articles (\url{https://en.wikipedia.org/wiki/List_of_Wikipedias}, accessed: July 23rd, 2018).
Today, nearly 2000 bots exist in Wikipedia.
Wikipedia is an open socio-technical system that allows us to study phenomena we can't observe
elsewhere.
How are automated system influencing the Wikipedian community?
Are there bots that implement hard rules ("Code is law") that nobody intended
to have but end up in there anyway?

=============================



(Background: Wikipedia/Wikimedia universe <-- where?)
- scope: which languages am I going to analyse?
- some statistics: how many articles, editors, bots, guidelines that pots
  implement?
- socio-technical system
- MediaWiki


Bots
- def

\begin{comment}
\begin{itemize}
	\item Erläutern Sie kurz, in welchem Themenbereich Ihre Arbeit angesiedelt ist. Wo werden Sie einen Beitrag leisten?
	\item Das Ziel sollte es sein, den groben Kontext Ihrer Arbeit darzustellen.
\end{itemize}
\end{comment}

\subsection{Thematische Einordnung der Arbeit}
\noindent \emph{Welche Literatur ist relevant für diese Arbeit?}
\begin{comment}
\begin{itemize}
	\item Nutzen Sie Ihr vom Betreuer/in bereitgestellt Literatur und (!) nutzen Sie Google Scholar für Ihre weitere Recherche. Das HCC.lab forscht insbesondere in den Bereichen Computer Supported Cooperative Work and Human-Computer Interaction. Daher sind zwei Konferenzen mit den dazugehörigen Publikationen von besonderen Interesse: CHI (Annual SIGCHI Conference: Human Factors in Computing Systems) und CSCW (Computer Supported Cooperative Work). Diese Konferenzen und noch einige mehr können auf der SIGCHI Webseite gefunden werden \cite{sigchi:2015}. Die dazugehörigen Publikationen in der Digital Library von ACM \cite{acm:2015}.
	\item Bitte geben Sie die relevanten Inhalte der Artikel kurz wieder.
	\item Das Ausarbeiten von ausgewählter Literatur bzw. verwandten Arbeiten hilft Ihnen, Ihre Ziele im nachfolgenden Abschnitt klarer zu definieren. Daher ist eine Auseinandersetzung mit der Literatur von Beginn an notwendig, auch wenn es zu diesem Zeitpunkt noch nicht erschöpfend erfolgen kann.
\end{itemize}
\end{comment}

\cite{DanaherEtAl2017}
- background for algorithmic governance, overview topics, methods, .. for research
  Danaher et al.~\cite{DanaherEtAl2017} propose a framework for researching algorithmic governance systems, their effectiveness and legitimacy.
  They provide a ``detailed map of key research themes, questions and methods'' which seems to be particularly useful for the current research proposal.


\cite{Geiger2017}
- describes complex interplay between social and technical systems in governing wikipedia

  Geiger~\cite{Geiger2017} is just one example of numerous works on the societal implications of bots and algorithmic systems in general in Wikipedia's ecosystem.
  It addresses various aspects and consequences of algorithmic governance within Wikipedia, such as driving away newcomers through an overly complicated bureaucratic process which results in Wikipedia turning into a gated community where only experienced users can fully participate.

\cite{GeiHal2017}
- replication study of the paper "Even Good Bots Fight" that raised the theory
  that bots get into conflict/revert wars constantly
- introduce a very detailed open methodology, have a github and .. repository
  with their whole data processing pipeline
- operationalise conflict
- generate a data set
- analyse edit summaries, cluster bot activities accordingly
- 2 metrics: # of bot-pair-per-article reverts, time between reverts
- quantitative and qualitative analysis
- conclusion: only 0.something of all bot-bot-reverts are genuine conflict

trace ethnography: "seek to understand data in the context it has within a community of practice," "data to be a starting point for further contextualization and interpretation,"
operationalise conflict!
distinguish conflict from non-conflict in bot reverts
compile data set + meta data
background: wikipedia researchers + long-time contributors
findings: overwhelming majority of the reverts are routine collaborative work between bots
metrics: time between reverts; number of reverts per article for the same bot pair
in addition: in depth analysis of random examples from different parts of the statistical distribution of the metrics
classify patterns in edit summaries

research question: "to what extent are bot-bot reverts
in Wikipedia genuine conflicts where disagreements about how Wikipedia ought to be written were
embedded in opposing bot codebases, versus cases like Shoreland’s Addbot that reflect the opposite?"

\cite{GeiRib2011}
- define the trace ethnography methodology
- traditional ethnography + heavy reliance on documentation
- related to ANT
- idea: reverse documentation traces to reconstruct the story (what happened
  there)
- ethical concerns: people couldn't have consented in an informed manner


\cite{HalKitRied2011}
Don't bite the newbies

\cite{IliadisRusso2016}
Introduction to Critical Data Studies + Literature Overview

big data is not neutral but instead consists of already constituted assemblages

"Data are a form of power."
"lack of data is another indication of power, the power
not to look or to remain hidden (Brunton and
Nissenbaum, 2015; Flyverbom et al., 2016)."
"foreground data’s power structures"

current research trends: positivist approach (and not critical)

Def:
"critical approach to Big Data
investigates meta-theoretical modes of conversation
and styles of scientific thinking"
"analyzes the
ground upon which positivistic Big Data science stands."
"How do Big Data inflect and interact with society,"
"CDS has emerged as a loose knit group of
frameworks, proposals, questions, and mani-
festos"
"What need to be established are long-term projects"

\cite{MuellerBirn2014}
  Mueller-Birn et al.~\cite{MuellerBirn2014} provide an overview over different tasks bots in Wikipedia carry out.
  They categorise these in two main groups: community services and guidelines/policies.
  The authors also suggest that attitudes towards bots have changed over time, and acceptance of bots has gradually grown.

\subsection{Zielstellung}
\noindent \emph{Welche Ziele werden mit der Arbeit verfolgt? Welche zentralen Fragen lassen sich daraus ableiten?}
Thema 1: Analyzing the translation of social into algorithmic rules

Our goal is to better understand how existing social rules are 
translated into algorithm manifestations by analyzing the existing 
source code but also by interviewing existing developers of bots and 
understanding their considerations when developing the bot code. We plan 
to unpack the “full socio-technical assemblage of algorithms” (Kitchin, 
2017). Based on the data collected, we identify all bots that disclose 
their code on their user pages in a first step (accountability), and 
select those that specifically active in the community sphere. We will 
identify up to five exemplary cases and will contextualize these cases 
by data from additional sources such as community pages (e.g. Village 
Pump). Furthermore, we plan to contact the operators of these bots (most 
often equals with the developers of the respective bot) and carry out 
interviews to better understand their motivations for developing the 
bots. Besides analyzing concrete examples of bots, we will also analyze 
the bot development frameworks (e.g. pywikibot) and compare the provided 
features with the bot activities. Our aim is to understanding the 
availability of specific source code might have influenced the influence 
of bots in the Wikimedia community. The result of this work packages is 
a better understanding of how social rules are translated into 
algorithmic rules, and a set of recommendations of ethical code 
guidelines for algorithmic systems that can be for example included in 
existing bot development frameworks. Furthermore, these ethical code 
guidelines can even be used in other platform or can inform platform 
design principles.

* Define Research question!
  * "What kinds of work are these bots were doing?
  * What were the bot’s developers intending the bots to do?" vgl GeiHal2017
\begin{comment}
\begin{itemize}
	\item Die Ziele sollten so spezifisch wie möglich sein. Das hilft Ihnen im Verlauf der Umsetzung zu prüfen, ob Sie Ihre Ziele erreichen konnten. Bitte achten Sie darauf, dass die gesetzten Ziele realistisch sind und das Sie in der Lage sind, das erfolgreiche Erreichen dieser Ziele im Bereich Evaluation zu prüfen.
\end{itemize}
\end{comment}


Ziele sind (lookup Claudia's email):
\begin{itemize}
    \item evaluate bot community engagement
    \item analyse bot frameworks (pywikibot) and their influence on prevalence of automated tasks
    \item guideline: ethical bot implementation

\end{itemize}
The purpose of this research is to determine whether bots end up implementing guidelines and norms in a distorted way, effectively creating constraints that no one wanted to have to begin with and turning them into the status quo.



\subsection{Geplante Vorgehensweise}
\noindent \emph{Welche einzelnen Aktivitäten müssen umgesetzt werden, um die Fragen zu beantworten und das Ziel der Arbeit zu erreichen?}
\begin{comment}
\begin{itemize}
	\item Aus den Fragen (vorheriger Abschnitt) können Sie dann Aktivitäten ableiten, die Ihnen helfen, Ihre weitere Arbeit zu strukturieren.
\end{itemize}
\end{comment}

Das sind keine Ziele, sondern Methodologie!
\begin{itemize}
  \item browse Wikipedia (which language versions? are bots language specific?); identify interesting bots
  \item study source code
  \item try talking to bots' developers
  \item study documentation/discussions around the bots development
\end{itemize}

\begin{itemize}
  \item \emph{document analysis}: review relevant guidelines and policies such as the Manual of style and Criteria for speedy deletion, as well as policies for creating bots such as the Bot policy and the Requests for approval
  \item \emph{code analysis}: analyse code of previously selected bots that seem to contradict a policy or a guideline in some way
  \item \emph{interviews with bot developers}: talk to bot developers about their intentions and approach when they conceived and implemented a particular bot
\end{itemize}

\emph{Bots}

This is a preliminary list of potentially interesting bots to look into:

\begin{itemize}
  \item HagermanBot (EN, there's a german equivalent): signs unsigned discussion entries -- there was a
  controversy, because some users specifically didn't want their posts to be
  signed
  \item AvicBot: moves pages from category A to category B, if merge request exist on Categories for Discussion, or removes a category entirely if the request is to empty the category
  \item CSDWarnBot: warns users when pages they've created are nominated for speedy deletion
  \item Mr.Z-bot: notifies users directly for users that are most likely spammers
  \item ClueBot NG + HBC AIV Helperbots(fighting vandalism)
  \item What about the bots that tweet about anonyme edits from inside the Parliament/Congress/Bundestag?
\end{itemize}

\subsection{Methodische Umsetzung}
\noindent \emph{Mit welcher Methodik setzen Sie die einzelnen Aktivitäten in Ihrer Arbeit um?}
Bereich: Data/Community Analytics <--- me?

trace ethnography
ANT

\begin{comment}
\begin{itemize}
	\item Eine wesentliche Grundlage der wissenschaftlichen Arbeit, ist die systematische Anwendung einer Forschungsmethode. Die erlaubt Ihnen ein systematisches Vorgehen bei Ihrer Masterarbeit.
	\item Am HCC.lab werden Arbeiten in den folgenden Bereichen geschrieben:
	       \begin{itemize}
				\item Information Retrieval
				\item Recommender Systeme
				\item Data/Community Analytics <--- me?
				\item Information Visualization
				\item Interaction Design
			\end{itemize}
	\item Sie sollten in der Lage sein, Ihre Arbeit in einen dieser Bereiche einzuordnen. Falls Sie Probleme dabei haben, wenden Sie sich bitte an Ihre/n Betreuer/in.
	\item Für die Auswahl der geeigneten Forschungsmethode verweise ich Sie auf zwei Bücher hin, die eine Zusammenstellung der verfügbaren Methoden und ihre praktischen Anwendung im Bereich Human-Computer Interaction diskutieren \cite{olson2014ways}, \cite{lazar2010research}. Beide Bücher sind bei Frau Prof. Müller-Birn verfügbar. Einzelne Artikel könnten Ihnen auch in einer elektronischen Version zur Verfügung gestellt werden.
\end{itemize}
\end{comment}

\subsection{Technische Umsetzung}
\noindent \emph{Mit welchen softwaretechnischen Hilfsmitteln soll die Arbeit realisiert werden?}
\begin{comment}
\begin{itemize}
	\item Selbstverständlich können Sie an der Stelle noch nicht alles wissen, aber Sie sollen sich hier bereits einen guten Überblick verschaffen.
\end{itemize}
\end{comment}

hm; no idea?
What I'm doing:

* talking to people <-- evtl record conversations;
* review source code
* read documentation; forums; mailing lists;
* charts: python; R; gnuplot
* assemble bot users
* assemble bot users for which the source code is published
--> python
* propose own bot and go through all the process

* methods:
  * begin with some data dumps, but ask additional questions and supplement with different data and methods
  * cooking data with care: take note of data science pipeline: explore the data, compile it carefully, ask critical questions
  * interviews with bot developers: understanding their considerations when developing the bot code.
\subsection{Erster Terminplan}
\noindent  \emph{Wie ist der generelle Zeitplan der Arbeit? }

Aug 2018 - Sep 2018 : identify interesting bots; literature review; expose anfertigen
Oct 2018 expose vorstellen
Oct 2018 - Dec 2018: Feedback von der Exposevorstellung einarbeiten; refine
Jan 2019: anmelden
Feb 2019: start writing
Jun 2019: abgeben

\begin{comment}
\begin{itemize}
	\item Sie sollten bereits wissen, wann Sie fertig sein wollen und von dort mit der Rückwärtsterminierung starten.
	\item Ihre Arbeit ist ein Projekt, daher planen Sie es auch wie eines. Nutzen Sie zur Visualisierung ein Gantt-Chart.
\end{itemize}
\end{comment}

%---------------------------------------------------
%----- Bibliography
%---------------------------------------------------
\phantomsection
\addcontentsline{toc}{chapter}{Literatur}   % headline
\bibliographystyle{alpha}  % citation style
\bibliography{bibliography} % bib file


%---------------------------------------------------
%----- Appendices
%---------------------------------------------------
\begin{comment}
\newpage
\section*{Anhang I: Auszug Prüfungsordnung Master}
\label{sec:master}

\begin{figure}[!h]
	\centering
		\includegraphics[width=0.8\textwidth]{pics/Auszug_Master_Pruefungsordnung.pdf}
	\caption{Auszug Prüfungsordnung Master}
\end{figure}
\end{comment}

\end{document}
